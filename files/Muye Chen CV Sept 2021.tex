% LaTeX Curriculum Vitae Template
%
% Copyright (C) 2004-2009 Jason Blevins <jrblevin@sdf.lonestar.org>
% http://jblevins.org/projects/cv-template/
%
% You may use use this document as a template to create your own CV
% and you may redistribute the source code freely. No attribution is
% required in any resulting documents. I do ask that you please leave
% this notice and the above URL in the source code if you choose to
% redistribute this file.

\documentclass[letterpaper]{article}

\usepackage{hyperref}
\usepackage{geometry}

% Comment the following lines to use the default Computer Modern font
% instead of the Palatino font provided by the mathpazo package.
% Remove the 'osf' bit if you don't like the old style figures.
\usepackage[T1]{fontenc}
\usepackage[sc,osf]{mathpazo}

% Set your name here
\def\name{Muye Chen}

% Replace this with a link to your CV if you like, or set it empty
% (as in \def\footerlink{}) to remove the link in the footer:
\def\footerlink{}

% The following metadata will show up in the PDF properties
\hypersetup{
  colorlinks = true,
  urlcolor = black,
  pdfauthor = {\name},
  pdfkeywords = {economics, statistics, mathematics},
  pdftitle = {\name: Curriculum Vitae},
  pdfsubject = {Curriculum Vitae},
  pdfpagemode = UseNone
}

\geometry{
  body={7in, 9in},
  left=0.75in,
  top=1in
}

% Customize page headers
\pagestyle{myheadings}
\markright{\name}
\thispagestyle{empty}

% Custom section fonts
\usepackage{sectsty}
\sectionfont{\rmfamily\mdseries\Large}
\subsectionfont{\rmfamily\mdseries\itshape\large}

% Other possible font commands include:
% \ttfamily for teletype,
% \sffamily for sans serif,
% \bfseries for bold,
% \scshape for small caps,
% \normalsize, \large, \Large, \LARGE sizes.

% Don't indent paragraphs.
\setlength\parindent{0em}

\usepackage{setspace}

\usepackage{hanging}

% Make lists without bullets
\renewenvironment{itemize}{
  \begin{list}{}{
    \setlength{\leftmargin}{0.35em}
  }
}{
  \end{list}
}

\begin{document}

% Place name at left
%{\huge  \bf \name}

% Alternatively, print name centered and bold:
\centerline{\huge \name}
\centerline{Last updated: September 2021}


\vspace{0.25in}
  Department of Economics \hfill \href{https://muyechenecon.github.io/}{https://muyechenecon.github.io/} \\
  Cornell University \hfill \href{mailto:mc2636@cornell.edu}{\tt mc2636@cornell.edu}  \\
  Uris Hall 443  \\
  Ithaca, NY 14850

\iffalse
\begin{minipage}{0.45\linewidth}
  Department of Economics \\
  Cornell University \\
  Uris Hall 443 \\
  Ithaca, NY 14850
\end{minipage}
\hfill
\begin{minipage}{0.45\linewidth}
  \begin{tabular}{ll}
    \hfill Phone: & (607) 216-7238 \\
	\hfill Email: & \href{mailto:mc2636@cornell.edu}{\tt mc2636@cornell.edu} \\
    Homepage: & \href{http://www.stat-or.unc.edu/}{\tt http://www.stat-or.unc.edu/} \\
  \end{tabular}
\end{minipage}
\fi
%----------------------------------------------------------------------------------------------------------------------
\section*{Education}
\begin{itemize}
	\item Ph.D. Economics, Cornell University \hfill Expected May 2022
	\item M.A. Economics, Cornell University\hfill2019
	\item M.S. Statistics, University of Illinois at Urbana-Champaign\hfill2016
	\item B.A. (\textit{magna cum laude}), Economics and Mathematics, University of Illinois at Urbana-Champaign\hfill2014
	\item Coursework toward B.S. in Finance, Shandong University at Weihai \hfill 2010 - 2012
\end{itemize}

%----------------------------------------------------------------------------------------------------------------------
\section*{References}
%\vspace{10pt}
\begin{minipage}{0.35\linewidth}
	Nicholas Sanders (chair) \\
	Assistant Professor \\
	Cornell University \\
	njsanders@cornell.edu
\end{minipage}
%\vspace{10pt}
\begin{minipage}{0.35\linewidth}
	Douglas Miller \\
	Professor \\
	Cornell University \\
	douglas.l.miller@cornell.edu
\end{minipage}
%\vspace{10pt}
\begin{minipage}{0.35\linewidth}
	Ivan Rudik \\
	Ruth and William Morgan \\
	Assistant Professor \\
	Cornell University \\
	irudik@cornell.edu
\end{minipage}

%----------------------------------------------------------------------------------------------------------------------
\section*{Fields of Study}
\begin{itemize}
\item Primary: Environmental Economics
\item Secondary: Labor Economics, Health Economics, Applied Microeconomics
\end{itemize}

%----------------------------------------------------------------------------------------------------------------------
\section*{Research}

\subsection*{Working Papers}
\begin{itemize}
\item \textbf{"Information Determines how Local Amenity Shocks Impact Labor Markets: News and the Differential Effects of Inland Oil Spills" (Job Market Paper)}
%\textbf{"Environmental Shocks, Information, and Local Labor Markets: Evidence from Inland Oil Spills" (Job Market Paper)}

%Leave or Stay: Labor Market Effects of Environmental Shocks when Information is (not) Available

%Labor Market Effects of Environmental Shocks when Information is (not) Available: Evidence from Inland Oil Spills

%When Environmental Shocks are (not) Publicized: Labor Market Effects of Inland Oil Spills

This paper provides the first causal estimates on how inland oil spills, one major type of environmental disasters, affect local labor markets. By exploiting severe inland petroleum oil spills and their news coverage status, I find that oil spills negatively affect county-level labor markets only when a spill is in the news. In the five years after a news-covered severe inland spill, employment and wage, which are the equilibrium quantity and price, and the number of establishments and labor force, which act as imperfect measures of labor demand and supply, all decrease significantly. When a severe inland spill is not reported in the news, the county-level labor markets are intact. The mechanism behind the observed effects is that spill information induces composition changes in gross out- and in-migration, which weakens the labor market conditions in sectors of low tradability. Back-of-the-envelope calculations suggest that, comparing to the control group, counties with news-covered spills lost 515K jobs, whose monetary value is equivalent to \$55.1B, and \$30.3B in foregone wages in aggregate in the post period.


\item \textbf{"Mortality Effects of Pollution when Information is (not) Missing: Evidence from Inland Oil Spills"}
%\textbf{"Environmental Shocks, Information, and Mortality: Evidence from Inland Oil Spills"}

%When Pollution Information is (not) Missing: Mortality Effects of Inland Oil Spills

Exploiting county-level variation in exposure to severe inland oil spills and their news coverage status, I estimate that oil spills raise ambient air pollution levels and mortality rates, but only when a spill is not reported in news. The rises in mortality rates are caused by the elevated level of air pollution and are concentrated in the most susceptible groups, children and the elderly. If a spill is covered in news, there are not only no changes in ambient air quality, but also persistent decreases in county-level mortality rates. By exploring heterogeneous effects, I show that the mortality rate decreases are due to out-migration. The findings imply that information on environmental disasters is beneficial to the environment and human health.

%This paper provides the first causal estimates on how oil spills, one major type of environmental disasters, affect mortality. Exploiting severe inland petroleum oil spills and their news coverage status, I find that oil spills raise ambient pollution levels and mortality rates only when a spill is not reported in news. If a spill is covered in news, there are not only no changes in ambient air quality, but also persistent decreases in county-level mortality rates. By exploring heterogeneous effects, I show that the mortality rate decreases are due to out-migration. The findings imply that information on environmental disasters is beneficial to the environment and human health. %Back-of-the-envelope calculations suggest that the aggregate mortality cost of spills without news coverage is about \$3.49 billion. Since the mortality rate decreases in counties with news-covered spills, the decreases reflect a general equilibrium effect and do not generate real savings. Suppose counties with not-news-reported spills are good counterfactuals for news-covered spill counties, the avoided mortality cost from air pollution in news-covered spill counties sums up to \$3.24 billion.
\end{itemize}

\subsection*{Work in Progress}
\begin{itemize}
\item \textbf{"Local Multipliers of Green Jobs"}
\end{itemize}

\subsection*{Publications}
\begin{itemize}
\item Chen, M., M. Regenwetter, and C. P. Davis-Stober. 2021. \textbf{"Collective Choice May Tell Nothing About Anyone’s Individual Preferences,"} \textit{Decision Analysis}, 18(1):1-24.
\item Allen, T. E., M. Chen, J. Goldsmith, N. Mattei, A. Popova, M. Regenwetter, F. Rossi, and C. Zwilling. 2015.\textbf{"Beyond Theory and Data in Preference Modeling: Bringing Humans into the Loop,"} In: Walsh T. (eds)  \textit{Algorithmic Decision Theory}, New York, NY: Springer. (Non-Peer Reviewed)
\end{itemize}

%----------------------------------------------------------------------------------------------------------------------
\section*{Research Assistance Experience}
\begin{itemize}
\item Research Assistant for Jura Liaukonyte, Cornell University\hfill2018
\item Research Assistant for Tatyana Deryugina, University of Illinois at Urbana-Champaign\hfill 2014-2016
\item Research Assistant for Michel Regenwetter, University of Illinois at Urbana-Champaign\hfill 2014-2016
\end{itemize}
%----------------------------------------------------------------------------------------------------------------------
\section*{Teaching Assistance Experience}

\begin{itemize}
\item ECON 3120 Applied Econometrics, George Jakubson, Cornell University\hfill Fall 2021
\item ECON 3120 Applied Econometrics, Doug McKee, Cornell University\hfill Spring 2020
\item PAM 6090 Empirical Strategies for Policy Analysis (Ph.D. level), Doug Miller, Cornell University\hfill Fall 2018, 2019
\item AEM 1300 Macroeconomics Theory and Policy, Arnab Basu, Cornell University\hfill Summer 2019
\item ECON 4250 Economics of Crime and Corruption, Daria Bottan, Cornell University\hfill Spring 2019
\item ECON 1110 Introductory Microeconomics, Jenny Wissink, Cornell University\hfill  Spring 2018
\item ECON 1110 Introductory Microeconomics, Stephanie Thomas, Cornell University\hfill Fall 2017
\end{itemize}

%----------------------------------------------------------------------------------------------------------------------
\section*{Honors, Awards, and Fellowships}
\begin{itemize}
\item Sage Fellowship, Cornell University\hfill 2016-2021
\item University Honors Scholar (highest recognition for undergraduate excellence), UIUC \hfill Spring 2014
\item Richard Winkel 2013 Convocation Speaker GPA Award, UIUC \hfill Spring 2014
\item First Place in the AXIS Student Challenge, UIUC \hfill Spring 2014
\item Elizabeth R. Bennett Scholarship in Mathematics, UIUC \hfill Spring 2014
\item Robert L. \& Amelia Louise Rivers Scholarship in Economics, UIUC \hfill Spring 2013
\item Merit-Based Scholarship, Shandong University \hfill Fall 2010 - Spring 2012
\end{itemize}
%----------------------------------------------------------------------------------------------------------------------
\section*{Conference and Seminar Presentations}
\begin{itemize}
\item Joint Labor Economics \& Applied Economics and Policy Workshop, Cornell University \hfill 2021
\item Northeast Workshop on Energy Policy and Environmental Economics, UPenn (Virtual) \hfill 2021
\item Sustainable Environment, Energy and Resource Economics Research Seminar, Cornell University \hfill 2020, 2021
\item European Mathematical Psychology Group Annual Conference, Copenhagen, Denmark \hfill 2016
\item Foundations of Utility and Risk Conference, University of Warwick, Coventry, UK \hfill 2016
\item The 36th Society for Judgment and Decision Making Annual Conference, Chicago, IL \hfill 2015
\item The 4th International Conference on Algorithmic Decision Theory, Lexington, KY\hfill 2015
\item Theories and Methods in Judgment and Decision Making Research Summer School, Mannheim, Germany\hfill2015
\end{itemize}


%----------------------------------------------------------------------------------------------------------------------
\section*{Skills}
\begin{itemize}
\item Programming: Stata, R, Python, Julia, MATLAB, \LaTeX
\item Language: Chinese (native), English (fluent)
\end{itemize}


\bigskip
\iffalse
% Footer
\begin{center}
  \begin{footnotesize}
    Last updated: \today \\
    \href{\footerlink}{\texttt{\footerlink}}
  \end{footnotesize}
\end{center}
\fi


\end{document}